\documentclass[a4paper,10pt]{article}

\usepackage[latin1]{inputenc}
\usepackage[ngerman]{babel}
\usepackage[T1]{fontenc}
\usepackage[dvips]{graphicx}

\author{Cool Cats!}
\title{Relazione Progetto Tecnologie Web}
\date{2009-03-14}

\makeindex

\begin{document}

\maketitle

\section{Introduzione}
Questo documento descrive come vengono organizzati i dati persistenti del sito web.

\section{Tematiche}
Le tematiche vengono salvate sotto la directory \textit{tematiche/} contenente una sottodirectory per ogni tematica. Il nome della sottodirecotry coincide con il nome della tematica. La data di una tematica viene presa dalla data della sua  directory.
Le sottodirectory contengono un file index.xml che raccoglie commenti e caratteristiche generali della tematica, e una serie di file xml che rappresentano ognuno una singola soluzione, composta da un approfondimento, una lista di pro e di contro, le domande e i risultati del sondaggio.
\subsection{tematiche/nome-tematica/index.xml}
\begin{itemize}
 \item descrizione generale: testo
 \item commenti: lista di ennuple (nome utente, data, voto, commento)
\end{itemize}
\subsection{tematiche/nome-tematica/possibile-soluzione.xml}
Il nome di una soluzione, con sondaggio annesso, \`e il nome del file che lo contiene.
\begin{itemize}
 \item domanda: testo
 \item pro: lista di vantaggi
 \item contro: lista di svantaggi
 \item opzioni: lista di coppie (descrizione, numero voti)
\end{itemize}

\subsection{utenti.xml}
Base di dati degli utenti con un account.
\begin{itemize}
 \item username
 \item email
 \item hash md5 della password
\end{itemize}

\section{Eventi (news)}
La directory \textit{eventi/} contiene il file index.xml che descrive gli eventi in generale e contiene gli eventuali file html statici che descrivono un singolo evento.
 
\subsection{eventi/index.xml}
\begin{itemize}
 \item data evento
 \item data inserimento
 \item titolo
 \item breve descrizione
 \item link alla pagina contenente l'evento (pu\`o essere un link esterno oppure interno)
\end{itemize}

\end{document}
